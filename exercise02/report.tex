\documentclass[a4paper]{article}

\usepackage{fancyhdr}

\pagestyle{fancy}
\fancyhf{}
\lhead{Optik und BOS: Wochenaufgabe 3}
\rhead{\today}
\lfoot{HTWG Konstanz}
\rfoot{Seite \thepage}

\begin{document}
	\thispagestyle{empty}
	
	\begin{center}\strut
		\bfseries\Huge
		Wochenaufgabe 3
	\end{center}
	\vfill
	
	\begin{center}\strut
		\textbf{Optik und bildgebende optische Systeme}\\
		Hochschule Konstanz Technik, Wirtschaft und Gestaltung\\
		Wintersemester 2021/22
	\end{center}
	%\vfill
	
	\begin{center}\strut
		\textbf{Team 6:}\\
		Milan Kaiser\\
		Ruwen Kohm\\
		Christian Schmeißer\\
	\end{center}
	\vfill
	\vfill

	\clearpage
	
	\section{Notwendigkeit von Linsensystemen}
	\textbf{Warum benötigt eine Kamera überhaupt ein Linsensystem für die Abbildung? Ein einfaches
		Loch genügt doch auch?}\\
	Obwohl mit einer einfachen Lochkamera auch eine Abbildung möglich wäre, haben Linsensysteme zahlreiche Vorteile. So ist zum Beispiel die Öffnungspupille bei einer Lochkamera sehr klein, wodurch nur wenig Licht auf den Sensor fällt. Außerdem können Linsensysteme mit bestimmten optischen Eigenschaften gebaut werden (Beispiel: telezentrische Objektive, bestimmte Brennweiten, etc.).\\
	
	\section{Auswahl des Objektives}
	\textbf{Warum haben die folgenden Eigenschaften einen Einfluss auf die Auswahl eines Objektivs?
		Und welchen Einfluss haben sie.}\\
	\begin{itemize}
		\item \textbf{Sensorgröße:} Das Objektiv muss in der Lage sein, über den gesamten Sensor abzubilden. Idealerweise reicht das projizierte Bild etwas über die Ecken des Sensors hinaus, um eine Abschattung (Vignettierung) zu vermeiden. Außerdem weisen Linsensysteme im Randbereich häufig stärkere Aberrationen, sowie Unschärfe auf als im Zentrum.\\
		\item \textbf{Pixelgröße:} Auch Linsensysteme weisen eine Auflösungsgrenze auf. So nützt ein Sensor mit kleineren Pixeln nichts, wenn die optische Qualität des Objektives nicht ausreicht um das Bild so hoch aufzulösen.\\
		\item \textbf{Objektauflösung:} Die Auflösung des Objektives (und respektive des Sensors) muss ausreichend sein um alle Details des Objektes aufzunehmen. Soll beispielsweise die Qualität eines bedruckten Produktes inspiziert werden, so muss das Auflösungsvermögen der Kamera (Objektiv + Sensor) zu der des Motivs passen.\\
	\end{itemize}

	\newpage

	\section{Objektivmontierung}
	\textbf{Machen Sie sich mit den verschiedenen Mounts vertraut. Welche werden wo vorwiegend
		eingesetzt?}\\
	\begin{itemize}
		\item \textbf{S-Mount:} ist ein besonders kleiner Gewindedurchmesser, für Objektive mit kleiner Baugröße. Anwendung in Platinenkameras, Handscanner, Überwachungskameras und Webcams.\\
		\item \textbf{C-Mount:} ist die Standardmontierung für Objektive im industriellen Umfeld. Der C-Mount besteht aus einem 1 Zoll Gewinde, welchen einen Abstand von 17,526mm zwischen Objektivrücken und Bildebene sicherstellt. Einsatz meist für Bildverarbeitung in Industriekameras.\\
		\item \textbf{CS-Mount:} ist eine kürzere Version des C-Mount. Ein C-Mount Objektiv kann mit einem 5mm Distanzring an diesem Mount angebracht werden. Andersherum funktioniert das nicht, da das Objektiv dann nicht fokussieren kann. CS-Mount-Objektive sind aufgrund ihres kurzen Auflagemaßes ideal, um preiswert extreme Weitwinkelobjektive zu konstruieren. Dies ist in der Überwachungstechnik besonders wichtig\\
		\item \textbf{M42:} wurde schon 1949 bei der Contax S von Zeiss verwendet. Diese Montierung, bestehend aus einem 42mm Gewinde, kommt also aus der Fotografie und bietet eine hohe Stabilität und Vibrationsresistenz.\\
		\item \textbf{Consumer-Brand:} Hin und wieder finden auch Objektive von Consumer-Brand Kameras wie Nikon oder Canon ihren Einsatz im Computervision Bereich. Diese verfügen meißt über einen Bayonettverschluss, welcher zum Beispiel gegen Eindringen von Staub oder Wasser abgedichtet werden kann.\\
	\end{itemize}
	
	\newpage

	\section{Objektiveigenschaften}
	\textbf{Zwei ganz wichtige Kenngrößen von Objektiven sind}\\
	\begin{itemize}
		\item \textbf{Brennweite:} Die Brennweite des Objektives gibt den Abstand zwischen Brennpunkt und Linsenebene an. Bei gleicher Sensorgröße führt eine größere Brennweite auch zu einer größeren Abbildung des Objekts. Umgangssprachlich verbindet man mit der Brennweite gerne den Zoom (variable Brennweite).\\
		\item \textbf{Lichtstärke:} Die Lichtstärke des Objektives ist abhängig von der Pupillenöffnung und wird durch die Blende begrenzt, beziehungsweise kontrolliert. Eine hohe Lichtstärke führt zu mehr eingesammelten Photonen, was zu einem helleren Bild führt.\\
	\end{itemize}


	\section{Lichtstarke Objektive}
	\textbf{Stimmt die Aussage: Es ist leichter ein lichtstarkes Weitwinkel-Objektiv herzustellen, als ein Teleobjektiv?}\\
	Ja, das stimmt.\\
	Die Lichtstärke eines Objektives erkennen wir an der Blendenzahl, auch f-Nummer genannt. Diese ergibt sich aus folgender Formel: $N = \frac{f}{D}$.\\
	Wobei N die Blendenzahl, f die Brennweite und D der Durchmesser der Eingangspupille ist. Bleibt die Brennweite gleich, wird aber der Durchmesser erhöht, so wird die Blendenzahl kleiner. Durch die größere Eingangspupille kann mehr Licht "gesammelt" werden, was bedeutet dass eine kleinere Blendenzahl für ein lichtstärkeres Objektiv steht. Bleibt nun der Durchmesser gleich und stattdessen die Brennweite erhöht, so erhöht sich auch die Blendenzahl. Das bedeutet ein Teleobjektiv (lange Brennweite), benötigt eine größere Eingangspupille als ein Weitwinkelobjektiv (kurze Brennweite) um die selbe Lichtstärke zu erreichen. Eine große Eingangspupille benötigt auch große Glaselemente im Linsensystem. Diese sind häufig sehr schwierig und teuer zu fertigen.\\
	
	\newpage 

	\section{Blende und Schärfe}
	\textbf{Stimmt die Aussage: Eine Verkleinerung der Blende (größere Zahl) erhöht immer die Schärfe meines Bildes?}\\
	Nein, stimmt nicht immer.\\
	Durch das Schließen der Blende wird die Tiefenschärfe erhöht, das heißt der Bereich welcher im Fokus ist wird vergrößert. Das erhöht die Schärfe im Bild, wenn mehrere Objekte abgebildet werden sollen, die unterschiedlich weit von der Kamera entfernt sind. Das Bild wird durch Abblenden nicht schärfer, wenn die Unschärfe durch eine qualitativ minderwertige Optik entsteht (Wobei es bei vielen Objektiven auch hier einen "Sweetspot" gibt, so dass Objektive bei Offenblende eine schlechtere Abbildungsleistung an den Tag legen).\\
	Ein zu starkes abblenden kann die Schärfe sogar verschlechtern, wenn bei besonders kleinen Blenden Beugungseffekte am Blendenrand eintreten (Diffraktion).\\
	
	
	\section{Zwischenringe}
	\textbf{Welchen Effekt haben Zwischenringe?}\\
	Zwischenringe erhöhen den Abstand zwischen Objektiv und Sensor. Durch diesen erhöhten Abstand kann das Objektiv näher Fokussieren. Leider ist es dann meistens nicht mehr möglich auf weit entfernte Objekte zu Fokussieren. Durch Zwischenringe wird der Abbildungsmaßstab erhöht, was aber auch dazu führt, dass weniger Licht auf den Sensor fällt. Für Inspektionsaufgaben bei denen der Abstand zwischen Produkt und Kamera sich nicht oder nur geringfügig ändert, ist der Einsatz von Zwischenringen eine kostengünstige Alternative zu Objektiven mit besserer Naheinstellgrenze.\\
	\textbf{Frage: Wird ein Objektiv für einen kleinen Sensor, mit Zwischenring an einem größeren Sensor verwendet (so dass kein Licht verloren geht), leidet dann die Abbildungsqualität?}\\
	
	\newpage
	
	\section{MTF}
	\textbf{Versuchen Sie in wenigen Sätzen zusammenzufassen was eine MTF ist:}\\
	\begin{itemize}
		\item \textbf{Warum misst man sie:} Die Modulations Transferfunktion (MTF) beschreibt die optische Qualität eines Linsensystems. Die MFT gibt Auskunft darüber, wie fein ein Objektiv Strukturen noch Kontrastreich übertragen kann. Auf der senkrechten Achse findet sich eine Prozentangabe (meist zwischen 0 und 1) die angibt wie nah die Abbildungsleistung des gemessenen Objektives an der eines perfekten Objektiv (100\% Lichtdurchlass)ist.
		In der Horizontalen ist der Abstand zum Mittelpunkt angegeben. Da Linsensysteme meistens zum Rand an Abbildungsleistung verlieren, sind deren MTF Charts fallende Kurven.\\
		\item \textbf{Wie misst man sie:} Betrachtet wird ein Testchart mit zunehmend feiner werdender Strukturierung von schwarz-weißen Balken (Linienpaare pro Millimeter) mit einem idealen, maximalen Objektkontrast von 100\%. Ein MTF-Diagramm trägt den gemessenen Bildkontrast gegenüber Linienpaaren /mm auf. Je mehr Linienpaare pro mm unterschieden werden können, desto besser ist die Auflösung des Objektivs.\\
		\item \textbf{Was kann man ablesen:} Das Diagramm zeigt gemessenen Bildkontrast gegenüber Linienpaaren/mm an. Das Diagramm gibt Auskunft über die Abbildungsleistung des Objektives.\\
		\item \textbf{Wo werden die lp/mm gemessen? Wie bestimmt man sie:} Anhand eines Testchartes mit zunehmend feiner werdender Strukturierung von schwarz-weißen Balken (Linienpaare pro Millimeter). Bei der Messung wird dann überprüft welche Feinheit an Linien pro Millimeter noch erkennbar sind. \\
	\end{itemize}
	
	\newpage
	
	\section{Optische Fehlabbildungen}
	\textbf{Schreiben Sie eine Tabelle mit den wichtigsten Fehlern für eine gute Abbildung.}\\
		\begin{tabular}{ c|c|c }
			Art des Fehlers & Ursache & Maßnahme \\ 
			\hline
			Defektaberration & Defektes Linsenelement & Objektiv umtauschen \\
			Chromatische Aberration & Verschiedene Brechung von $\lambda$ & Korrekturlinsen und/oder Software\\
			Sphärische Aberration & Lichtstrahlen nicht im selben Fokus & Asphärische Linsen\\
			Räumliche Verzerrung & zu kurze Brennweite & längere Brennweite\\
			Vignettierung & Mechanische Abschattung & kleinerer Sensor oder Software
		\end{tabular}

	
	\section{Objektivarten}
	\textbf{Erstellen Sie eine Tabelle für folgende Objektivtypen.}\\
		\begin{tabular}{ c|c|c|c }
			Objektiv & Vorteile & Nachteile & Einsatz \\ 
			\hline
			Festbrennweite & Oft Lichtstark & Kein Zoom & Industrielle Anwendung, Fotografie\\
			Verklebte Linsen & Günstig, gute Qualität & keine Anpassung möglich & Industrielle Anwendung\\
			Zoomobjektiv & Variable Brennweite & oft Lichtschwach & Fotografie, CCTV
		\end{tabular}
	
	\section{Telezentrische Objektive}
	\textbf{Wann verwenden Sie Telezentrische Objektive?}\\
	Wenn eine perspektivische Verzerrung die Inspektion erschwert. Nicht jede perspektivische Verzerrung kann durch affine Transformation in Software ausgeglichen werden. Daher werden telezentrische Objektive eingesetzt wenn zum Beispiel Maßstäbe bestimmt werden müssen.\\
	
	\section{Schwarz-Weiß-Kamera}
	\textbf{Kann man mit einer Schwarz-Weiß-Kamera Farbbilder generieren?}\\
	Ja, zum Beispiel kann das Objekt nacheinander mit verschiedenen Farben beleuchtet werden. Bei jeder Farbe wird ein Bild gemacht, wodurch wieder Informationen für einzelne Farbkanäle vorliegen.

	
	
\end{document}
